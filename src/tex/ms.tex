% Define document class
\documentclass[twocolumn]{aastex631}
\usepackage{showyourwork}

%% Include packages
\usepackage{amsmath}
\usepackage{calc}
%% Custom commands
\DeclareMathOperator*{\argmax}{argmax} % in your preamble
\newcommand{\AR}{active region}
\newcommand{\dem}{$\mathrm{EM}(T)$}
\newcommand{\vlos}{v_{\mathrm{LOS}}}
\newcommand{\aiapy}{\texttt{aiapy}}
\renewcommand{\sectionautorefname}{Section}
\renewcommand{\subsectionautorefname}{Section}
\renewcommand{\subsubsectionautorefname}{Section}

% Begin!
\begin{document}

% Title
\title{Constraining Heating Properties in Long, Steady Loops in Active Region NOAA }

% Author list
\author[0000-0001-9642-6089]{W. T. Barnes}
\affiliation{National Research Council Postdoctoral Research Associate Residing at the Naval Research Laboratory, 4555 Overlook Avenue SW, Washington, DC 20375, USA}
\affiliation{NASA Goddard Space Flight Center, Greenbelt, MD, USA}
\affiliation{American University, Washington, DC, USA}
\author[0000-0001-6102-6851]{H. P. Warren}
\affiliation{Space Science Division, Naval Research Laboratory, 4555 Overlook Avenue SW, Washington, DC 20375, USA}
\author[0000-0003-4739-1152]{J. W. Reep}
\affiliation{Space Science Division, Naval Research Laboratory, 4555 Overlook Avenue SW, Washington, DC 20375, USA}
\correspondingauthor{W. T. Barnes}
\email{will.t.barnes@nasa.gov}

% Abstract with filler text
\begin{abstract}
Observations of coronal loops at the periphery of active regions have been found to be steady, lasting much longer than a radiative cooling time, and near-isothermal at approximately 1.5 MK.
These relatively steady and high-density structures cannot be explained by either hydrostatic loop simulations or impulsive nanoflare heating.
In this paper, we analyze \textit{Hinode} EIS observations of these loop structures and measure both densities and emission measure distributions.
Additionally, we analyze the time variability of these structures using SDO AIA observations and measure both the emission measure distributions as well as the cross-correlations between channel pairs.
We then model the hydrodynamic evolution of these loop structures for several different heating scenarios, include steady uniform heating, stratified footpoint heating, and impulsive heating.
From our model results, we derive density and temperature diagnostics in order to constrain the parameter space of feasible heating models using our observations.
We find that..
\end{abstract}
\keywords{Sun:corona, Sun:UV radiation, methods:data analysis, hydrodynamics}

% Main body with filler text
\section{Introduction}
\label{sec:intro}

Lorem ipsum dolor sit amet, consectetuer adipiscing elit.
Ut purus elit, vestibulum ut, placerat ac, adipiscing vitae, felis.
Curabitur dictum gravida mauris, consectetuer id, vulputate a, magna.
Donec vehicula augue eu neque, morbi tristique senectus et netus et.
Mauris ut leo, cras viverra metus rhoncus sem, nulla et lectus vestibulum.
Phasellus eu tellus sit amet tortor gravida placerat.
Integer sapien est, iaculis in, pretium quis, viverra ac, nunc.
Praesent eget sem vel leo ultrices bibendum.
Aenean faucibus, morbi dolor nulla, malesuada eu, pulvinar at, mollis ac.
Curabitur auctor semper nulla donec varius orci eget risus.
Duis nibh mi, congue eu, accumsan eleifend, sagittis quis, diam.
Duis eget orci sit amet orci dignissim rutrum.

Nam dui ligula, fringilla a, euismod sodales, sollici- tudin vel, wisi.
Morbi auctor lorem non justo, nam lacus libero, pretium at, lobortis vitae.
Donec aliquet, tortor sed accumsan bibendum, erat ligula aliquet magna.
Morbi ac orci et nisl hendrerit mollis, suspendisse ut massa, cras nec ante.
Pellentesque a nulla cum sociis natoque penatibus et magnis dis parturient.
Aliquam tincidunt urna, nulla ullamcorper vestibulum turpis.
Pellentesque cursus luctus mauris \citep{Luger2021}.

\bibliography{bib}

\end{document}
